%%%%%latex preamble%%%%%

\documentclass{article}
%% maxwidth is the original width if it is less than linewidth
%% otherwise use linewidth (to make sure the graphics do not exceed the margin)
\usepackage{alltt}
\usepackage[sc]{mathpazo}
\usepackage{amsmath, amsthm, amssymb}
\usepackage{graphicx}
\usepackage[T1]{fontenc}
\usepackage{geometry}
%\geometry{verbose,tmargin=2.5cm,bmargin=2.5cm,lmargin=1.5cm,rmargin=1.5cm}
\setcounter{secnumdepth}{2}
\setcounter{tocdepth}{2}
\usepackage{url}
\usepackage[unicode=true,pdfusetitle,
  bookmarks=true,bookmarksnumbered=true,bookmarksopen=true,bookmarksopenlevel=2,
breaklinks=false,pdfborder={0 0 1},backref=false,colorlinks=false]
{hyperref}
\hypersetup{pdfstartview={XYZ null null 1}}
\usepackage{float}
\usepackage{bm}
\usepackage{tikz}
 %changes default sectioning commands 
%\usepackage{breakurl}
\renewcommand{\thesubsection}{(\alph{subsection})}
\renewcommand{\thesubsubsection}{\roman{subsection}.}
\usepackage{lastpage}
\usepackage{fancyhdr}
\pagestyle{fancy}
\lhead{}
\chead{\leftmark}
\rhead{}
\lfoot{Aaron Gonzales; Algorithms}
\cfoot{Homework 1}
\rfoot{Page \thepage\ of \pageref{LastPage}}
\IfFileExists{upquote.sty}{\usepackage{upquote}}{}

\begin{document}

\title{Homework 1, CS591, Fall 2014}
\author{Aaron Gonzales}
\maketitle


given the following vectors:

\begin{align}
	x = \{1, 3, 5, 7, 6, 8, 5, 2\}  \\
	y = \{7, 11, 9, 6, 4, 3, 1, 3\}  \\
	z = \{3, 5, 7, 9, 8, 7, 6, 5\}  
\end{align}

\section{wavelet transform each vector}

An un-normalized 8-point Haar matrix \[H_8\] is shown below

\[ H_{8} =  
\begin{bmatrix} 
	1&1&1&1&1&1&1&1 \\ 
	1&1&1&1&-1&-1&-1&-1 \\
	1&1&-1&-1&0&0&0&0& \\ 
	0&0&0&0&1&1&-1&-1 \\ 
	1&-1&0&0&0&0&0&0& \\
	0&0&1&-1&0&0&0&0 \\ 
	0&0&0&0&1&-1&0&0& \\ 
	0&0&0&0&0&0&1&-1 
\end{bmatrix}
\]
and will be used to multiply each vector.

\subsection{vector x}

\[ 
	\begin{bmatrix}
		1 \\
		3 \\
		5 \\
		7 \\
		6 \\
		8 \\
		5 \\
		2 
	\end{bmatrix}
		*  
\begin{bmatrix} 
	1&1&1&1&1&1&1&1 \\ 
	1&1&1&1&-1&-1&-1&-1 \\
	1&1&-1&-1&0&0&0&0& \\ 
	0&0&0&0&1&1&-1&-1 \\ 
	1&-1&0&0&0&0&0&0& \\
	0&0&1&-1&0&0&0&0 \\ 
	0&0&0&0&1&-1&0&0& \\ 
	0&0&0&0&0&0&1&-1 
\end{bmatrix}
=
\begin{bmatrix}
	1 + 3 + 5  + 7 + 6 + 8 + 5 + 2 \\
	1 + 3  + 5 + 7 - 6 - 8 -5 -2 \\
	1 + 3 + 5 + 7 \\
	6+ 8 -5 -2 \\
	1-3 \\
	5-7 \\
	6-8 \\
	5-2
\end{bmatrix}
\]
\[
	= \{ 37, -4, -8, 7, -2, -2, -2, 3 \}
\]





\subsection{vector y}

\[ 
	\begin{bmatrix}
		7 \\
		11 \\
		9 \\
		6 \\
		4 \\
		3 \\
		1 \\
		3 
	\end{bmatrix}
		*  
\begin{bmatrix} 
	1&1&1&1&1&1&1&1 \\ 
	1&1&1&1&-1&-1&-1&-1 \\
	1&1&-1&-1&0&0&0&0& \\ 
	0&0&0&0&1&1&-1&-1 \\ 
	1&-1&0&0&0&0&0&0& \\
	0&0&1&-1&0&0&0&0 \\ 
	0&0&0&0&1&-1&0&0& \\ 
	0&0&0&0&0&0&1&-1 
\end{bmatrix}
=
\begin{bmatrix}
	7+11+9+6+4+3+1+3 \\
	7+11+9+6 -4 -3 -1 -3 \\
	7+11 -9-6 \\
	4+3 -1 -3 \\
	7-11 \\
	9-6 \\
	4-3 \\
	1-3

\end{bmatrix}
\]
\[
	= \{ 44, 22, 3, 3, -4, 3, 1, -2 \}
\]



\subsection{vector z}


\[ 
	\begin{bmatrix}
		1 \\
		3 \\
		5 \\
		7 \\
		6 \\
		8 \\
		5 \\
		2 
	\end{bmatrix}
		*  
\begin{bmatrix} 
	1&1&1&1&1&1&1&1 \\ 
	1&1&1&1&-1&-1&-1&-1 \\
	1&1&-1&-1&0&0&0&0& \\ 
	0&0&0&0&1&1&-1&-1 \\ 
	1&-1&0&0&0&0&0&0& \\
	0&0&1&-1&0&0&0&0 \\ 
	0&0&0&0&1&-1&0&0& \\ 
	0&0&0&0&0&0&1&-1 
\end{bmatrix}
=
\begin{bmatrix}
	3 + 5 + 7 + 9 + 8 + 7 + 6 + 5 \\
	3 + 5 + 7 + 9 + -9 + -7 + -6 + -5 \\
	3 + 3 + -7 + -9 \\
	8 + 7 + -6 + -5 \\
	3 + -5 \\
	7 + -9 \\
	8 + -7 \\
	6 + -5
\end{bmatrix}
\]
\[
	= \{ 50, -2, -8, 4, -2, -2, 1, 1 \}
\]



\section{fourier transform each vector}
The transformation \[W\] of size \[N\times N \] can be defined
as 
\[W = \left(\frac{\omega^{jk}}{\sqrt{N}}\right)_{j,k=0,\ldots,N-1} \]
, or equivalently:

\[ W = \frac{1}{\sqrt{N}}  
\begin{bmatrix}
	1&1&1&1&\cdots &1 \\
	1&\omega&\omega^2&\omega^3&\cdots&\omega^{N-1} \\
	1&\omega^2&\omega^4&\omega^6&\cdots&\omega^{2(N-1)}\\
	1&\omega^3&\omega^6&\omega^9&\cdots&\omega^{3(N-1)}\\
	\vdots&\vdots&\vdots&\vdots&\ddots&\vdots\\
	1&\omega^{N-1}&\omega^{2(N-1)}&\omega^{3(N-1)}&\cdots&\omega^{(N-1)(N-1)}\\
\end{bmatrix}
\]
\[ \omega = e^{-\frac{2\pi i}{N}} \] 


\section{minkowski distance}

\subsection{raw vectors}


\subsection{wavlet coeff.}


\subsection{largest three wavelet coefficients of x}


\subsection{largest three wavelet coefficients of y}


\subsection{which one of the above approximations matches the ordering of the distances from 
the raw vectors?}












\end{document}
