%%%%%latex preamble%%%%%
\documentclass[titlepage]{article}\usepackage[]{graphicx}\usepackage[]{color}
%% maxwidth is the original width if it is less than linewidth
%% otherwise use linewidth (to make sure the graphics do not exceed the margin)
\makeatletter
\def\maxwidth{ %
  \ifdim\Gin@nat@width>\linewidth
  \linewidth
  \else
  \Gin@nat@width
  \fi
}
\makeatother


\usepackage{minted}
\usepackage{listings}
\definecolor{bg}{rgb}{0.95,0.95,0.95}
\definecolor{mygreen}{rgb}{0,0.6,0}
\definecolor{mygray}{rgb}{0.5,0.5,0.5}
\definecolor{mymauve}{rgb}{0.58,0,0.82}
\lstset{ %
  backgroundcolor=\color{white},   % choose the background color; you must add \usepackage{color} or \usepackage{xcolor}
  basicstyle=\footnotesize,        % the size of the fonts that are used for the code
  breakatwhitespace=false,         % sets if automatic breaks should only happen at whitespace
  breaklines=true,                 % sets automatic line breaking
  captionpos=b,                    % sets the caption-position to bottom
  commentstyle=\color{mygreen},    % comment style
  deletekeywords={...},            % if you want to delete keywords from the given language
  escapeinside={\%*}{*)},          % if you want to add LaTeX within your code
  extendedchars=true,              % lets you use non-ASCII characters; for 8-bits encodings only, does not work with UTF-8
  frame=single,                    % adds a frame around the code
  keepspaces=true,                 % keeps spaces in text, useful for keeping indentation of code (possibly needs columns=flexible)
  keywordstyle=\color{blue},       % keyword style
  morekeywords={*,...},            % if you want to add more keywords to the set
  numbers=left,                    % where to put the line-numbers; possible values are (none, left, right)
  numbersep=5pt,                   % how far the line-numbers are from the code
  numberstyle=\tiny\color{mygray}, % the style that is used for the line-numbers
  rulecolor=\color{black},         % if not set, the frame-color may be changed on line-breaks within not-black text (e.g. comments (green here))
  showspaces=false,                % show spaces everywhere adding particular underscores; it overrides 'showstringspaces'
  showstringspaces=false,          % underline spaces within strings only
  showtabs=false,                  % show tabs within strings adding particular underscores
  stepnumber=2,                    % the step between two line-numbers. If it's 1, each line will be numbered
  stringstyle=\color{mymauve},     % string literal style
  tabsize=2,                       % sets default tabsize to 2 spaces
  title=\lstname                   % show the filename of files included with \lstinputlisting; also try caption instead of title
}
\usepackage{alltt}
\usepackage[sc]{mathpazo}
\usepackage{amsmath, amsthm, amssymb}
\usepackage{graphicx}
\usepackage[T1]{fontenc}
\usepackage{geometry}
\geometry{verbose,tmargin=2.5cm,bmargin=2.5cm,lmargin=1.5cm,rmargin=1.5cm}
\setcounter{secnumdepth}{2}
\setcounter{tocdepth}{2}
\usepackage{url}
\usepackage[unicode=true,pdfusetitle,
  bookmarks=true,bookmarksnumbered=true,bookmarksopen=true,bookmarksopenlevel=2,
breaklinks=false,pdfborder={0 0 1},backref=false,colorlinks=false]
{hyperref}
\hypersetup{pdfstartview={XYZ null null 1}}
\usepackage{float}
\usepackage{bm}
\usepackage{tikz}
 %changes default sectioning commands -> 1,a, etc.
%\usepackage{breakurl}
\usepackage{lastpage}
\usepackage{fancyhdr}
\pagestyle{fancy}

%%% Header and Footer %%% 
\lhead{}
\chead{\leftmark}
\rhead{}
\lfoot{Aaron Gonzales; Data Mining}
\cfoot{Assignment 3}
\rfoot{Page \thepage\ of \pageref{LastPage}}
\IfFileExists{upquote.sty}{\usepackage{upquote}}{}

\begin{document}

\title{Assignment 3, CS591, Fall 2014}
\author{Aaron Gonzales}
\maketitle



\section{Local outlier factor}
Discussion with other classmates led to analyzing the threshold for lof as
dictated by the assignment (2). Using the DMwR pacakge in R to run LOF on our
dataset, we can see that a thresold of 2 excludes the entire dataset. Figure
\ref{fig:lofscores} shows the plot of scores and Listing \ref{lst:lofr} shows
the R output. 

\begin{listing}
  \begin{minted}[bgcolor=bg]{r}
data <-
read.table('/cs/datamining/data-mining-is-fun/assignments/two/data.txt',
    header=FALSE)
data <- scale(data)

  library(DMwR)
  lof <- lofactor(data=dat, k= 10)
> sum(lof > 1.0)
  [1] 8599
> sum(lof > 1.3)
  [1] 636
> sum(lof > 1.31)
  [1] 521
> sum(lof > 1.4)
  [1] 0
\end{minted}
  \label{lst:lofr}
  \caption{R implementation of LOF}
\end{listing}

\begin{figure}
  \label{fig:lofscores}
  \centering
  \includegraphics[width=0.40\linewidth]{lofscore.pdf}
  \caption{LOF scores with k=10. No item has an LOF score higher than 1.4}
\end{figure}

\section{LOF code}
The MATLAB code for the algorithm and analysis is below in Listings
\ref{lst:lof} and \ref{lst:script} on page \pageref{lst:lof}; a plot showing the relationship
between $k$ and the number of outliers is show in Figure \ref{fig:outliers} on
page \pageref{fig:outliers}. As the number of neighbors $k$ increases, the
number of outliers decreases, as expected. 

%\lstinputlisting[language=Matlab]
%{../../lof_ag.m} \label{lst:lof} 
\begin{listing}
  \inputminted[bgcolor=bg]{matlab}{../../lof_ag.m}
\label{lst:lof}
\caption{Local Outlier Factor function}
\end{listing}

\begin{listing}[language=Matlab]
  \begin{minted}[bgcolor=bg]{matlab}
k=[2 5 10 20 30 40 50 60 70 80 90 100];
threshold=1.3;
data=dlmread('/cs/datamining/data-mining-is-fun/assignments/two/data.txt');
data=zscore(data);
%% lof_ag takes data, number of neighbors to try, and a threshold LOF score 
lof_ag(data, k, 1.3)/
\end{minted}
\caption{Local outlier data preparation}
\label{lst:script}
\end{listing}

\begin{figure}
  \label{fig:outliers}
  \centering
  \includegraphics[width=0.85\linewidth]{outlierplot.pdf}
  \caption{Number of outliers vs $k$ using a threshold LOF score of 1.3}
\end{figure}

Other methods were tried to increase the speed of the algorithm, though a KD
tree didn't seem to help much due to the high dimensionality of our data.
Running time to process the set with the $k$ vector in Listing \ref{lst:script}
was approximately 30 minutes, owing heavily to MATLAB's efficient $k$NN search
functionality.


\end{document}
